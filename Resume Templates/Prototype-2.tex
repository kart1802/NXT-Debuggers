\documentclass{article}
\usepackage[utf8]{inputenc}
\usepackage[T1]{fontenc}
\usepackage{amsmath}
\usepackage{amssymb}
\usepackage{enumitem}
\usepackage{geometry}
\usepackage{fancyhdr}

\usepackage{vwcol} 
\usepackage{multicol}
\usepackage{nopageno}
\usepackage{fontawesome}
\usepackage{setspace}
\usepackage{helvet}
 \fontfamily{phv}\selectfont
\usepackage[compact]{titlesec}
\titlespacing{\section}{0pt}{0pt}{0pt}
\usepackage{xcolor}
\usepackage{hyperref}
\hypersetup{
colorlinks = true,
linkcolor = blue,
urlcolor = red
}
\geometry{
 a4paper,
 total={190mm,257mm},
 left=10mm,
 top=25mm,
 headheight=8mm
 }

\urlstyle{same}

\pagestyle{fancy}
\fancyhf{}
\chead{\LARGE {\uppercase{\textbf{Prahlad Amudan}} }}



\begin{document}
\begin{multicols}{2}
%\begin{vwcol}[widths={0.4,0.6},
%sep=.8cm, justify=flush,rule=0pt,indent=1em]
%[
%\section{\LARGE{\uppercase{\textbf{Prahlad Amudan}}}}
%All human things are subject to decay. And when fate summons, Monarchs must obey.
%]

\section*{\large{\textcolor{blue}{\uppercase{contact:}}}} 
%\vspace{3pt}
\flushleft{

\faPhone \hspace{1mm} Phone: +919833004100\\
\faEnvelopeO \hspace{1mm} {\color{red}\underline{\href{mailto:prahlad2001a@gmail.com}{prahlad2001a@gmail.com}}} \\
\faHome \hspace{1mm} 2nd Floor,Brahma Niwas,Near Pangong Tso Lake, Chembur,Mumbai-400089 \\ 
\faLinkedin\hspace{1mm}  LinkedIn: {\color{red}\underline{\href{mailto:prahlad2001a@gmail.com}{www.linkedin.com/in/Prahlad-Amudan}}}\\
\faGithub\hspace{1mm} Github: {\color{red}\underline{\href{mailto:prahlad2001a@gmail.com}{www.github.com/in/Prahlad-Amudan}}}
}

\vspace{10pt}
\section*{\large{\textcolor{blue}{\uppercase{education:}}}}

%\vspace{3pt}
\renewcommand{\labelitemii}{$\circ$}
\flushleft{

{\textbf{Veermata Jijabai Technological Institute(VJTI)}}\\{\textbf{CGPA}}: 10 \hspace{5mm} {\textbf{Percentage}}: - \hspace{5mm} {\textbf{From}}: 2019 - 2023\\
{\textbf{Course Work}} : Thermodynamics, Fluid Mechanics, Coding, Mechatronics, Entreprenuership,Computer Programming, Basic Electrical Engineering, Elements of Mechanical Engineering, Mechanics, Graphics.\\


{\textbf{\underline{Professional Associations}}}:
\begin{itemize}[noitemsep,nolistsep]
	
	\item {\textbf{Member of Entrepreneurship-cell VJTI}}: Event coordinator for national level business planning competition
	
\end{itemize}

}
\vspace{5pt}


\section*{\large{\textcolor{blue}{\uppercase{Skills:}}}}

%\vspace{3pt}
\begin{flushleft}
\begin{itemize}[noitemsep,nolistsep]
	\item AutoCAD, Solidworks, AutoCAD Mechanical, Autodesk Inventor
	\item Excel, C++, HTML, Javascript, Python
	\item Management, Leadership, Organization, Public Speaking, Problem-solving, Teamwork
\end{itemize}
\end{flushleft}
\vspace{5pt}

\section*{\large{\textcolor{blue}{\uppercase{Hobbies:}}}}

%\vspace{3pt}
\begin{flushleft}
\begin{itemize}[noitemsep,nolistsep]
	\item Reading Different kinds of books, Programming, listening to classical Indian music
	\item Blogging, Volunteering, Traveling, Art, Design, Music, Reading, Video Gaming.
\end{itemize}
\end{flushleft}
\vspace{5pt}
\section*{\large{\textcolor{blue}{\uppercase{objective:}}}}

%\vspace{3pt}
\begin{flushleft}
The decision about what to put into your paragraphs begins with the germination of a seed of ideas; this “germination process” is better known as brainstorming. There are many techniques for brainstorming; whichever one you choose, this stage of paragraph development cannot be skipped. Building paragraphs can be like building a skyscraper: there must be a well-planned foundation that supports what you are building.
\end{flushleft}

\vspace{3pt}
\columnbreak

\section*{\large{\textcolor{blue}{\uppercase{projects:}}}}

%\vspace{3pt}

\begin{enumerate}[noitemsep,nolistsep]
	\item {\textbf{Resume Builder}}\\
	\hfill {\textbf{From}}: May-20 {\textbf{To}} July-20\\
	Piranhas rarely feed on large animals; they eat smaller fish and aquatic plants. When confronted with humans, piranhas’ first instinct is to flee, not attack. Their fear of humans makes sense. Far more piranhas are eaten by people than people are eaten by piranhas. If the fish are well-fed, they won’t bite humans.
	\item {\textbf{Fan Controlled using Bluetooth}}\\
	\hfill {\textbf{From}}: Dec-19 {\textbf{To}} Mar-20\\
	Although most people consider piranhas to be quite dangerous, they are, for the most part, entirely harmless. Piranhas rarely feed on large animals; they eat smaller fish and aquatic plants. When confronted with humans, piranhas’ first instinct is to flee, not attack. Their fear of humans makes sense. Far more piranhas are eaten by people than people are eaten by piranhas. If the fish are well-fed, they won’t bite humans.
\end{enumerate}

\vspace{5pt}

\section*{\large{\textcolor{blue}{\uppercase{Internship:}}}}

%\vspace{3pt}

\begin{enumerate}[noitemsep,nolistsep]
	\item {\textbf{Internship at Google}}\\
	\hfill {\textbf{From}}: Mar-20 {\textbf{To}} May-20\\
	Developing writers can often benefit from examining an essay, a paragraph, or even a sentence to determine what makes it effective. On the following pages are several paragraphs for you to evaluate on your own, along with the Writing Center's explanation.
	\item {\textbf{Internship at Lockheed Martin}}\\
	\hfill {\textbf{From}}: May-20 {\textbf{To}} Nov-20\\
	 Scientists' research has revealed that viruses are by far the most abundant life forms on Earth. There are a million times more viruses on the planet than stars in the universe. Viruses also harbor the majority of genetic diversity on Earth. Scientists are finding evidence of viruses as a planetary force, influencing the global climate and geochemical cycles. They have also profoundly shaped the evolution of their hosts. The human genome, for example, contains 100,000 segments of virus DNA.
\end{enumerate}

\vspace{5pt}

\section*{\large{\textcolor{blue}{\uppercase{achievements:}}}}

%\vspace{3pt}
\begin{flushleft}
\begin{itemize}[noitemsep,nolistsep]
	\item Was a part of State level football team
	\item Won many national and international level olympiads
	\item having a soft corner towards art
\end{itemize}
\end{flushleft}
\vspace{5pt}

%\end{vwcol}
\end{multicols}
\end{document}